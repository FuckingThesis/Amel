%----------------------------------------------------------------------------------------
%	Glossary
%----------------------------------------------------------------------------------------
\newglossaryentry{recuit}{
    name={recuit simulé},
    description={Protocole de recherche de la forme la plus stable du système étudié. Cette forme est dans ce casElle de plus basse énergie correspondant à la recherche du minimum global de la surface d'énergie potentielle du système moléculaire. Cette méthode a une phase de simulation à haute température permettant d'explorer l'espace conformationnel et une deuxième de refroidissement pour atteindre un autre minimum local ou global du système.}
}
\newglossaryentry{coupscal}{
    name={couplage scalaire},
    description={Interaction correspondant à un transfert d'aimantation entre deux noyaux séparés par une ou plusieurs liaisons covalentes. La force est mesurée par les constantes de couplage qui dépendent du type des noyaux impliqués.}
}
\newglossaryentry{coupdip}{
    name={couplage dipolaire},
    description={Association entre deux noyaux proches dans l'espace. Le transfert d'aimantation observés entre les deux noyaux est affecté par les mouvements browniens en solution. Cet effet est communément nommé  dans cette situation l'effet Overhauser nucléaire (\acrshort{noe}).},
    plural={couplages dipolaires}
}
\newglossaryentry{cdr}{
    name={contrainte de distance ambiguë},
    description={Chaque pic croisé d'un spectre \acrshort{noesy} est traité comme étant la superposition des signaux  des assignations possibles du pic. Les contraintes calculées (Distance effective $R$) sont proportionnelles à la somme de l'inverse des distances interatomiques élevées ($r$) à la puissance 6 de chaque assignation possible. \begin{equation}\frac{1}{R^6} = \sum_{i}^{n}\frac{1}{r_{i}^6}\end{equation}}
}
\newglossaryentry{rdc}{
    name={couplage dipolaire résiduel},
    description={Dans le cas d'expérimentations \acrshortpl{rmn} dans des milieux partiellements orientés, des couplages dipolaires résiduels sont mesurés et leur valeur est proportionnelle à $\frac{1}{r^3}$ ($r$ est la distance inter-atomique) mais dépend également des angles $\theta$ entre les liaisons covalentes et le vecteur du champ magnétique statique. }
}
\newglossaryentry{apidef}{
    name={interface de programmation (API)},
    description={Regroupe l'ensemble des classes, méthodes et fonctions d'une application souvent stockés dans une bibliothèque logicielle et qui permettent aux autres logiciels d'avoir accès aux différentes fonctionnalités de l'application. Elle correspond par exemple aux classes d'ARIA sauvegardant les paramètres.}
}
\newglossaryentry{framew}{
    name={framework},
    description={Ensemble cohérent de composants logiciels permettant de définir les bases d'une application. Il définit un environnement de travail au développeur proposant un ensemble d'outils et guide l'architecture du futur logiciel.}
}
\newglossaryentry{ormdef}{
    name={mapping objet relationnel (ORM)},
    description={Concept de programmation permettant de convertir un ensemble d'objets en une base de données relationnelle.}
}

%----------------------------------------------------------------------------------------
%	Acronyms
%----------------------------------------------------------------------------------------
\newacronym{xml}{XML}{Extensible Markup Language}
\newacronym{noesy}{NOESY}{Spectroscopie d'Effet Overhauser Nucleaire}
\newacronym{aria}{ARIA}{Ambiguous Restraints for Iterative Assignment}
\newacronym{rmn}{RMN}{Résonance Magnétique Nucléaire}
\newacronym{cns}{CNS}{Crystallography \& NMR System}
\newacronym{mtf}{MTF}{Molecular Topology File}
\newacronym{noe}{NOE}{Effet Overhauser Nucléaire}
\newacronym{embl}{EMBL}{European Molecular Biology Laboratory}
\newacronym{bis}{BIS}{Bioinformatique Structurale}
\newacronym{wenmr}{WeNMR}{Worldwide e-Infrastructure for NMR and structural biology}
\newacronym{api}{API}{Interface de Programmation (Application Programming Interface)}
\newacronym{cib}{CIB}{Centre d'Informatique pour la Biologie}
\newacronym{ccpn}{CCPN}{Collaborative Computing Project for the NMR community}
\newacronym{pdb}{PDB}{Protein Data Bank}
\newacronym{mtv}{MTV}{Modèle Template Vue}
\newacronym{mvc}{MVC}{Modèle Vue Contrôleur}
\newacronym{html}{HTML}{Hypertext Markup Language}
\newacronym{orm}{ORM}{Mapping Objet Relationnel}
\newacronym{rmsd}{RMSD}{Root Mean Square Deviation}
\newacronym{dry}{DRY}{Don't Repeat Yourself}
\newacronym{cgi}{CGI}{Common Gateway Interface}
\newacronym{casd}{CASD-NMR}{Critical Assessment of Automated Structure Determination of Proteins from NMR Data}