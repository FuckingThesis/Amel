% !TEX root = ../thesis-example.tex
%
\pdfbookmark[0]{\abstractname}{\abstractname}

\begingroup

\let\clearpage\relax
\let\cleardoublepage\relax
\let\cleardoublepage\relax

\chapter*{\abstractname}\label{sec:abstract}
\vspace*{-10mm}

Dans la fibrose kystique (FC), l'inflammation est détectée au début de la voie aérienne, même avant le début de l'infection bactérienne. Cela suggère que les mécanismes autres que l'infection sont à l'origine du processus initial d'inflammation. Parmi ces processus, il y a le stress oxydatif. Ce dernier est largement reconnu comme un élément essentiel de plusieurs maladies.
Récemment, nous avons observé que les cellules et les tissus des patients bronchiques des FC présentent une augmentation de la concentration de Cu, dans la production de radicaux libres, dans les activités enzymatiques pro et antioxydantes, et une diminution de la concentration de glutathion, un chelate naturel de Cu intracellulaire. Ces résultats et d'autres nous ont permis d'établir le lien avec Cu, le stress oxydatif, l'inflammation et l'infection.
En étudiant le niveau d'expression d'un certain nombre de gènes codant pour les protéines de l'homéostasie du Cu, nous avons constaté que l'expression de la protéine prion cellulaire (PrPC) a été modifiée. PrPC est une glycoprotéine chantée de glycosyl phosphatidyl inositol (GPI) qui a été impliquée dans la propagation et l'agrégation des prions dans le système nerveux central qui conduit à des maladies des encéphalopathies spongiformes transmissibles (EST).
Cependant, en dépit de plusieurs études in vitro et in vivo qui ont démontré la capacité de PrPC à interagir avec d'autres protéines, à lier le cuivre (Cu) avec une affinité élevée et à protéger les cellules contre le stress oxydatif, ses fonctions physiologiques sont encore à l'étude, en particulier dans Des tissus extra-neuronaux, tels que l'épithélium bronchique.
Dans le présent projet, nous avons étudié le rôle de PrPC dans l'architecture cellulaire pulmonaire en déterminant son impact sur l'intégrité des jonctions épithéliales pulmonaires par rapport à l'inflammation, l'homéostasie du Cu, le stress oxydatif et l'infection. Dans ce cas, je présenterai seulement les résultats de la première partie de mon projet concernant le rôle physiologique de PrPC dans les cellules épithéliales bronchiques normales. En utilisant la PCR quantitative et l'immunoblotting, nous avons démontré que l'ARNm de PrPC et la protéine PrPC mature sont exprimés dans des cellules bronchiques épithéliales humaines (HBE et A549). Nous avons également démontré par immunofluorescence que le PrPC est localisé dans les domaines apicaux et latéraux où il se localise localement avec des jonctions de protéines adhérentes et désmosomes. La localisation du domaine PrPC dépend du niveau de polarisation de la cellule. Ces données ont été confirmées par la co-immunoprécipitation montrant une interaction de la protéine PrPC avec une protéine de jonction adhérente désmosome. En outre, nous avons montré que le traitement au Cu augmente l'expression de PrPC aux niveaux de l'ARNm et de la protéine, confirmant son rôle dans la protection cellulaire contre le stress oxydatif généré sur l'excès de cuivre. Enfin, nous avons démontré que l'invalidation du gène PrPC dans les cellules HBE diminue la résistance trans-épithéliale (TER), un indicateur de la qualité de la barrière de jonction.
Dans l'ensemble, nos résultats soulignent l'importance de PrPC dans le contrôle du contact cellulaire à cellulaire dans les cellules épithéliales pulmonaires normales et suggèrent que sa déréglementation pourrait affecter la barrière de jonction dans des maladies telles que la fibrose kystique.

\vspace*{20mm}

\renewcommand{\abstractname}{Abstract}
{\usekomafont{chapter}\abstractname}\label{sec:abstract-diff} \\

In cystic fibrosis (CF), inflammation is detected early in the airways, even before the onset of bacterial infection. This suggests that mechanisms other than infection are at the origin of the initial inflammation process. Among these processes, there is the oxidative stress. The latter is widely accepted as a critical component of several diseases. 
Recently, we observed that both cells and tissues from bronchial CF patients display an increase in Cu concentration, in free radicals production, in pro-and antioxidant enzyme activities, and a decrease in glutathione concentration, a natural intracellular Cu chelator. These results and others have allowed us to establish the link with Cu, oxidative stress, inflammation and infection.
While investigating the expression level of a number of genes encoding proteins of Cu homeostasis, we found that the expression of the cellular prion protein (PrPC) was altered. PrPC is a glycosyl phosphatidyl inositol (GPI)-anchored glycoprotein that have been involved in prion infection propagation and aggregation in the central nervous system that leads to transmissible spongiforme encephalopathies (TSE) diseases. 
However, despite several in vitro and in vivo studies that demonstrated the capacity of PrPC to interact with other proteins, to bind copper (Cu) with high affinity, and to protect cells against oxidative stress, its physiological functions are still under investigations, particularly in extra-neuronal tissues, such as bronchial epithelium. 
In the present project, we investigated the role of PrPC in the lung cellular architecture, by determining its impact on the integrity of the lung epithelial junctions in relation to inflammation, Cu homeostasis, oxidative stress, and infection.
Herein, I’ll only present the results of the first part of my project regarding the physiological role of PrPC in normal bronchial epithelial cells. Using quantitative PCR and immunoblotting, we demonstrated that the PrPC mRNA and the mature PrPC protein are expressed in human epithelial bronchial cells (HBE and A549). We also demonstrated by immunofluorescence that the PrPC is localized at the apical and lateral domains where it co-localizes with adherent and desmosomes protein junctions. The PrPC domain localization is dependent on the level of cell polarization. These data were confirmed by the co-immunoprecipitation showing an interaction of the PrPC protein with a protein of adherent desmosome junction. Moreover, we showed that Cu treatment increases PrPC expression at the mRNA and protein levels, confirming its role in the cellular protection from oxidative stress generated upon copper excess. Finally, we demonstrated that the invalidation of PrPC gene in HBE cells decreases the trans-epithelial resistance (TER), an indicator of the quality of the junctional barrier. 
Overall, our results stress the importance of PrPC in the control of cell to cell contact in normal lung epithelial cells and suggest that its deregulation might affect junction barrier in diseases such as cystic fibrosis.


\endgroup

\vfill