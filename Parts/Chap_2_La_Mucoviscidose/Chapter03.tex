% Chapter X

\chapter{Approches thérapeutiques fondamentales} % Chapter title


\label{ch:02-03} % For referencing the chapter elsewhere, use \autoref{ch:name} 

%----------------------------------------------------------------------------------------

% \section{}
Depuis la découverte de CFTR de nombreux progrès en termes de soin médicaux ont été fait, mais bien que l’espérance de vie soit d’environ une cinquantaine d’année en moyenne pour les patients, la mucoviscidose reste une maladie incurable, mais la question est, pourquoi ?
La première des raisons réside dans le fait que le gène, cause première de la maladie, n’a été découvert que très récemment. Depuis cette découverte de nombreux progrès en termes de soin médicaux ont été fait, d’ailleurs, Il existe deux approches thérapeutiques fondamentales. La première consiste à cibler les problèmes pulmonaires en s’attaquant aux infections microbiennes ou tenter de modifier la couche de mucus. La seconde consiste à cibler les causes premières de la maladie, c’est-à-dire refaire fonctionner la protéine CFTR défectueuse ou bien la remplacer par thérapie génique ou à l’aide d’autres moyens pharmacologiques. Les thérapies ciblant les causes premières sont le plus souhaitable à l’avenir pour le traitement de la mucoviscidose, mais de tels traitements sont difficiles à développer.
La première des approches thérapeutiques fondamentales consiste en des traitements multiples destinés à soulager les symptômes de la maladie, mais ils sont contraignants de par le fait qu’ils nécessitent un suivi quotidien. L’un des traitements essentiels de la mucoviscidose et aussi la plus contraignante est la kinésithérapie respiratoire qui permet de désencombrer les bronches, mais il faut l’effectuer plusieurs fois par jour pendant 1h environ. En complément sont prescrit au patient des antibiotiques, des anti-inflammatoires et des médicaments permettant de fluidifier le mucus afin de soulager les troubles respiratoires. Seulement ces traitements ne visent qu’à soigner les conséquences et non les causes de la maladie. C’est pourquoi actuellement la deuxième des approches thérapeutiques fondamentales est la plus investigué en recherchant un traitement curatif visant à identifier et à contrôler les actions des transporteurs comme CFTR causant la déshydratation du mucus qui est à l’origine de ces troubles. Comme décrit précédemment le dysfonctionnement de la protéine CFTR serait à l’origine de la maladie, il semble donc logique de penser qu’une correction de son activité soit la clé de la guérison pour les patients. C’est pourquoi de nombreux travaux ont été effectués afin de restaurer l’activité de CFTR directement ou indirectement et ce grâce à de nombreuses molécules ou par correction génétique de la mutation responsable. En effet la restauration de la sécrétion chlorure est l’enjeu majeur du développement de thérapeutiques pharmacologique de la mucoviscidose. A ce jour il est possible d’agir grâce à des composants pharmacologiques sur CFTR en ciblant son système de régulation ou directement sur le canal par l’intermédiaire d’activateurs ou de potentiateurs de CFTR directs ou indirects.


%------------------------------------------------

% \subsection{Subsection Title}

% Content