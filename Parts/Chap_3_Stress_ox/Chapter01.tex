% Chapter X

\chapter{Généralité} % Chapter title


\label{ch:03-01} % For referencing the chapter elsewhere, use \autoref{ch:name} 

%----------------------------------------------------------------------------------------

% \section{}

Lorsque la balance entre antioxydant et oxydant n’est plus capable de prévenir les altérations des fonctions physiologiques il se met en place ce que l’on définit d’état de stress oxydatif. 

Dans les voies respiratoires en particulier, ce que l’on appelle les ROS, terme générique qui regroupe nombre d’éléments et molécules très réactives, peuvent avoir des effets bénéfiques ou non en fonction de leurs concentrations. Tous organismes aérobies produisent des espèces réactives de l’oxygène (ROS) et de l’azote (RNS). Les ROS sont des espèces chimiques réactives contenant de l’oxygène radicalaire comme le radical superoxyde \textit{O}$_{2}$\up{-}, le radical hydroxyle (OH) et non radicalaire comme le peroxyde d’hydrogène (\textit{H}$_{2}$\textit{O}$_{2}$). Les ROS sont aussi bien produits de manière endogène qu’exogène. L’un des gros producteurs de ROS endogène est les complexes NADPH oxydase (NOX) présent dans les membranes cellulaires, la mitochondrie, les peroxysomes et le réticulum endoplasmique, il en existe 7 isoformes distinct. Polluants, tabac, radiation sont eux des producteurs exogènes. Les métaux tels que le fer, le cuivre, le chrome, le vanadium et le cobalt sont capables de faire un cycle redox dans lequel un seul électron peut être accepté ou donné par le métal. Cette action catalyse la production de radicaux et d'espèces réactives d'oxygène. La présence de ces métaux sous une forme non complexée peut augmenter considérablement le niveau de stress oxydatif. On pense que ces métaux induisent des réactions de Fenton et la réaction de Haber-Weiss, dans laquelle le radical hydroxyle est généré à partir du peroxyde d'hydrogène.
Les RNS sont une famille des molécules antimicrobiennes dérivées du monoxyde d'azote et du radical superoxyde produit respectivement via l’oxyde nitrique synthase (NOS) et NADPH oxydase.

La production de ROS est un processus physiologique nécessaire dans de nombreuses fonctions cellulaires. Plusieurs voies de signalisations sont régulées par les ROS parmi elles la voix des MAPK, JNK et de nombreuses autres. Une signalisation efficace et normale nécessite un déséquilibre court de la balance redox. Pourtant un stress oxydatif peut survenir lorsque qu’une augmentation de la production de ROS n’est pas finement régulée par les antioxydants et que celle-ci perdure dans le temps. 
Ainsi il a été montré qu’un excès de ROS peut favoriser la transcription de gènes pro-inflammatoire (Bartling and Drumm 2009)\cite{bartling_oxidative_2009} et parmi les dommages dû à un excès de ROS, l’oxydation peut causer des dommages irréversibles à de nombreuses cibles moléculaires notamment les lipides mais aussi l’ADN (Genestra 2007)\cite{genestra_oxyl_2007}.


%------------------------------------------------

% \subsection{Subsection Title}

% Content