% Chapter X

\chapter{Dans la Mucoviscidose} % Chapter title


\label{ch:03-02} % For referencing the chapter elsewhere, use \autoref{ch:name} 

%----------------------------------------------------------------------------------------

% \section{}

De nombreuses études ont été menées sur le stress oxydatif et son rôle dans la mucoviscidose 
Au niveau du plasma une faible concentration en vitamine E et sélénium a été observé. On retrouve aussi de nombreux marqueurs de l’oxydation des lipides. Ainsi qu’une diminution de la concentration en GSH circulant. Le plasma constitue le compartiment extra-cellulaire le plus large et l’état redox du plasma régule probablement de nombreux aspect de la fonction endothélial. \cite{jones_redox_2002}\cite{jones_cysteine/cystine_2004}\cite{imhoff_extracellular_2009}\cite{jiang_oxidant-induced_2005} (Jones et al. 2002)(Jones et al. 2004)(Imhoff and Hansen 2009)(Jiang et al. 2005)
Dans le plasma la concentration en GSH mais pas GSSG est réduite, de même pour la GSH mitochondriale. Comme la GSH est connu pour inhibé la dégradation de I$\kappa$$\alpha$, il est fort probable qu’une faible concentration cellulaire en GSH soit responsable de l’activation de Nf-kB et participe ainsi au maintiens de l’inflammation. 
De nombreux marqueurs du stress oxydatif comme la lipide hydroperoxydation ou l’oxydation des protéines ont été mesuré dans le plasma des patients CF et sont associées à une diminution de la concentration en antioxydant dans le plasma. Ainsi un niveau élevé de peroxydation des lipides peut être associé à des dysfonctionnements pulmonaire induisant à des dommages dans la structure membranaire. A été reporté aussi une augmentation de la concentration en peroxyde d’hydrogène intracellulaire.
Dans la cellule la balance rédox est principalement régulée par le niveau de glutathion. Le glutathion existe sous une forme monomérique réduite GSH et dimérique oxydé GSSG. La GSH est présente en forte concentration dans les fluides extra cellulaires, dans le poumon et dans les cellules. La GSH extra cellulaire neutralise les radicaux libre produit par les neutrophiles lors de la réponse inflammatoire. Sachant que comme précisé précédemment CFTR permet le transport de GSH entre les cellules et le milieu extra cellulaire, il est raisonnable de penser que le contenu en GSH intracellulaire peut être altéré dans la mucoviscidose.
Parmi les compartiments cellulaires les plus étudié dans la mucoviscidose on retrouve en première position la mitochondrie qui est la source la plus significative de ROS dans la cellule et a par conséquent fait l’objet de nombreuses études. Ceci a commencé dans les années 70 lorsqu’il a été reporté une forte consommation en oxygène mitochondrial chez les patients atteint de mucoviscidose en comparaison de personnes saines. Plus tard il a été montré que cette forte consommation d’oxygène diminuait après traitement par de la ouabaïne suggérant que cette augmentation de consommation d’oxygène est liée à une augmentation de l’activité de la NaK ATPase même si aucun défaut de l’activité NaK ATPase a été observé chez les patient CF. Malgré quelque controverse concernant un état stable de déséquilibre rédox dans la mucoviscidose, de nombreuse étude ont montré nombre d’évidence suggérant une augmentation du stress oxydant dans la mucoviscidose.
Une augmentation de l’urinary 8-hydroxydeoxyguanosine suggère une oxydation et des dommages à l’ADN. On observe une augmentation dans la peroxydation des lipides dans les poumons. Par contre il n’y a aucune différence dans le niveau de GSH dans les poumons. Par contre une diminution du niveau mitochondriale a été observé couplé à un niveau élevé d’oxydation de l’ADN mitochondriale et une diminution de l’activité aconitase résultant de son oxydation chez les patients CF. 
D’un point vu cellulaire en dehors de la mitochondrie il a été montré qu’il y a une augmentation de l’activité oxydase des granulocytes. Que les phagocytes présentes une fonction bactéricide diminué alors que l’on a une forte production de ROS dû à la diminution du transport de chlore dans les phagosomes. [PubMed:(Shapiro, Feigal, and Lam 1979)][PubMed: (Feigal and Shapiro 1979)][PubMed: (Stutts et al. 1986)][PubMed: (Turrens et al. 1982)][PubMed: (Schwarzer et al. 2007)] [PubMed:(Brown et al. 1995)][PubMed: (L. W. Velsor, van Heeckeren, and Day 2001)] [PubMed:(Gao et al. 1999)][PubMed: (Mangione et al. 1994)] [PubMed: (Leonard W. Velsor et al. 2006)]. [PubMed: (Kelly-Aubert et al. 2011)] [PubMed: (Berry and Brewster 1977)] [PubMed: (Painter et al. 2008)] \cite{shapiro_mitrochondrial_1979} \cite{feigal_mitochondrial_1979} \cite{stutts_oxygen_1986}\cite{turrens_effect_1982}\cite{schwarzer_organelle_2007}\cite{brown_oxidative_1995}\cite{velsor_antioxidant_2001}\cite{gao_abnormal_1999}\cite{mangione_erythrocytic_1994}\cite{velsor_mitochondrial_2006}\cite{kelly-aubert_gsh_2011}\cite{berry_granulocyte_1977}\cite{painter_role_2008}

Globalement, ces résultats indiquent que les niveaux élevés de ROS mitochondriales et cellulaires sont associés à un état déficitaire en CFTR. Les effets des ROS sont double chez les patients CF, d'une part, il est connu que le stress cellulaire induit par les ROS inhibe la maturation de CFTR (Rab et al. 2007)\cite{rab_endoplasmic_2007}.
D’autre part, une augmentation de ROS conduit à l’activation des voix de signalisation MAPK (Genestra 2007)\cite{genestra_oxyl_2007}. Cette cascade est connue pour réguler l’expression des gènes pro-inflammatoire dans les cellules (Verhaeghe et al. 2007)\cite{verhaeghe_role_2007}, il est donc logique de penser que les ROS sont impliqués dans l'initiation et / ou le maintien de l'inflammation lors d’un déficit en CFTR. En outre, un excès de la production de cytokines pro-inflammatoires peut augmenter la production de ROS (Ozben 2007)\cite{ozben_oxidative_2007} perpétuant le cercle vicieux de l’inflammation chez les patients CF.
Trois superoxyde dismutases (SOD) ont été décrit chez les mammifères, Cu / Zn-SOD ou SOD1, Mn-SOD SOD2 et SOD-extra-cellulaire ou SOD3 (Bowler and Crapo 2002)\cite{bowler_oxidative_2002}. Ces enzymes sont impliquées dans la diminution du niveau d’anion superoxyde qui endommagent les cellules à une concentration excessive (Bowler et al. 2004)\cite{bowler_extracellular_2004}. Des Modifications de l'expression et / ou de l'activité des SOD ont été décrits dans plusieurs pathologies telles que la sclérose latérale amyotrophique pour la SOD1 (Rosen et al. 1993)\cite{rosen_mutations_1993}, les cardiomyopathies pour la SOD2 (Robinson 1998)\cite{robinson_role_1998} et les maladies pulmonaires pour SOD3 (Smith et al. 1997)\cite{smith_reduced_1997}. En effet, la SOD3 est fortement exprimée dans les poumons et est associé à une diminution du recrutement de neutrophiles, suggérant un rôle important dans la régulation de l'inflammation pulmonaire (Bowler and Crapo 2002)\cite{bowler_oxidative_2002}.
Bien qu’il n’y a aucune preuve directe à l’implication de SOD3 chez les patients CF, son haut niveau d’expression pulmonaire n’exclut pas qu’il a probablement un rôle important à jouer chez les patients CF. En outre, les cytokines pro-inflammatoires augmente l'expression de la SOD3 en culture et dans les modèles animaux qui présentent des lésions pulmonaires (Bowler et al. 2003)\cite{bowler_evidence_2003}(Marklund 1992)\cite{marklund_regulation_1992}.
Globalement un défaut des systèmes antioxydant semble associé à toutes les mutations de CFTR. Et chez les patients CF une défense antioxydante inadéquate est associé à une augmentation du stress oxydatif qui contribue au déclin de la fonction pulmonaire.


%------------------------------------------------

% \subsection{Subsection Title}

% Content